\documentclass[12pt, titlepage]{article}
\usepackage{amssymb, amsmath, hyperref,setspace}
\usepackage[margin=1in]{geometry}
\usepackage{bbm}
\usepackage{color}
\usepackage{commath}
\DeclareMathAlphabet{\mathscr}{OT1}{pzc}{m}{it}
\usepackage{epsfig, graphicx}
\usepackage{verbatim}
\renewcommand{\v}[1]{\ensuremath{\mathbf{#1}}} % for vectors
\newcommand{\gv}[1]{\ensuremath{\mbox{\boldmath$ #1 $}}} 
% for vectors of Greek letters
\newcommand{\grad}[1]{\gv{\nabla} #1} % for gradient
\let\divsymb=\div % rename builtin command \div to \divsymb
\renewcommand{\div}[1]{\gv{\nabla} \cdot #1} % for divergence
\newcommand{\curl}[1]{\gv{\nabla} \times #1} % for curl
% for double partial derivatives
\title{Non-linear Pulse Propogation: Split-Step Style}
\author{Adam A. S. Green}
\begin{document}
\begin{spacing}{1.1}
\bibliographystyle{plain}
\maketitle
\begin{abstract}
This is the abstract
\end{abstract}

\section{Introduction}

Ultimately, the information is encoded onto photons. (How? Modulation? Different frequencies?). The interaction
of these photons and the fibre that carries them is completely described by Maxwell's equations. With most
fibers [cite corning smf], a linear description usually sufficies. 

However, if the pulse intensity becomes high enough, or if the fibre is specifically made, one can obtain
non-linear effects. 

This leads to a very rich field of the study of non-linear propogation of light, giving rise to a rich array
of phenomena: solitons, filimentation, Raman Scattering, Fibre Lasers, [find citations for them all]

In this paper, I will give a broad introduction to the field. 

\subsection{Linear and Nonlinear Effects}

In media, Maxwell's equations become modified with the introduction of the displacement field, $D = \epsilon_0 E + P$ , where $P$ is the induced polarization of the medium, and $E$ is the electric field. There is also a 
modification to the magnetic field in a similar way, but most of the material we will be studying have a
neglegible magnetic polarizability.

If the response of the material is linear, then we can write $P = \chi E$, and our Maxwell's equations are
once again, tractable.

(Show how the equations are changed).

In the presence of non-linearities, or large optical pulses (need to read Boyd about this), we Taylor expand
the response of the field:
$\v{P}$
\begin{equation}
\v{P}(t) = \epsilon_0  \left( \chi^{(1)} \v{E}(t) + \chi^{(2)} \v{E}^2(t) + \chi^{(3)} \v{E}^3(t) +...\right)
\end{equation}

Manipulating this into a wave equation form, we get:
\begin{equation}
\nabla^2 \v{E} - \frac{1}{c^2} \pd[2]{\v{E}}{t} = \mu_0 \pd[2]{\v{P}_L}{t} + \mu_0 \pd[2]{\v{P}_{NL}}{t}
\end{equation}
\begin{alignat}{2}
       \div{E} &= \frac{\rho}{\epsilon_0} &\qquad \div{B} &=0\\
       \curl{E} &=- \pd{B}{t}  & \curl{B} &= \mu_0 + \mu_0 \epsilon_0 \pd{E}{t} 
       \end{alignat}
By preforming some manipulations, these equations can be massaged into the familiar form of 
\begin{equation}
\pd{U}{\xi}= \frac{is}{2} \pd[2]{U}{t} + i N^2 |U|^2U 
\end{equation}
\end{spacing}
\end{document}


